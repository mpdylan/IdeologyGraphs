\documentclass[10pt]{amsart}
\usepackage{amsmath}
\usepackage{amssymb}
\usepackage{graphicx}
\usepackage{multicol}
\usepackage{tikz}

\title{}
\author{Dylan Murphy}
\institution{University of Arizona School of Information\footnote{1103 E. 2nd St., Tucson, Arizona, 85721, USA}}
\date{\today}

\begin{document}

\maketitle

\begin{abstract}
The propagation of informatic phenomena such as ideology, sentiment, and disinformation across social networks such as Facebook or Twitter has been studied mathematically and computationally by many scholars, including researchers at Facebook itself, in some detail over the past decade.
The expansion of social media as a marketing and political tool used by corporations, politicians, and state actors makes this a topic of practical social concern as well as a topic of mathematical interest.
Recently, Brooks and Porter\cite{HZBMAP2020} described a dynamical systems model for the evolution of ideology on directed social networks in the presence of media accounts.
Their model reproduced empirically observed effects such as the concentration of accounts around influential media sources and the formation of so-called ``echo chambers.''
In this article, we consider several expansions of the Brooks-Porter model.
\end{abstract}

\section{Introduction}



\section{Existing models}

In this model, a social network is represented by a directed graph.
The vertices of this graph represent accounts; the edges represent the follower relation.
Note that in real social networks, the relation may be asymmetric (e.g., Twitter, where accounts \emph{follow} one another) or symmetric (e.g., Facebook, where accounts typically are \emph{friends}).
Follower relationships are constant; after the follower relationships are established in the initialization of the simulation, they do not change.

subsection{Media accounts}

A key feature of the Brooks-Porter model is the presence of media accounts, which are accounts that do not follow any other accounts.
Clearly, media accounts are static ``sources'' of ideology; since they do not follow other accounts, their state variables remain constant in time.

\section{Expanded models}

\subsection{Dynamic follower updating}

The principal novel component of our model is a probabilistic step for updating the network structure.
This allows accounts to stop following other accounts (media and non-media) or to begin following other accounts.
In principle, this will allow us to observe echo chamber formation through the lens of follower relationships as well as ideological dimension.

\subsubsection{Unfollowing}

In our model, when a media account's content is rejected by a follower account for its ideology, there is a nonzero probability that that follower will furthermore unfollow the media account.
A similar unfollowing rule is in place for non-media accounts, but with a different weighting.
This is to incorporate the assumption that personal relationships are more resilient to differences in ideology than an individual's relationship with a media organization.\footnote{A weakness of this concept is that it treats all non-media follower relationships as ``personal relationships'', although on some networks (e.g., Twitter), many accounts follow highly influential non-media individuals with whom they have no personal connection (e.g., celebrities). The models we consider do not distinguish such high-influence but non-media individuals from other accounts.}

The follower updating rule is as follows. At each time step, if an account $i$ rejects the content of a media account $m$, the edge is deleted with probability
\[
p_u = \max\{1, s_M(\|x_i - x_m\| - c)}
\]
where $c$ is the receptiveness parameter, and $s_M$ is the \emph{media sensitivity}, a parameter controlling how easily individuals are driven to unfollow media accounts.

The rule for unfollowing individual accounts is the same, except with a different (generally much smaller) sensitivity parameter $s_I$ replacing $s_M$.

\subsubsection{Following}

We also include a mechanism for non-media accounts to begin following media accounts if presented with content sufficiently similar to their own ideology.
Since a non-follower cannot observe a media account directly, this is modeled as a second-order effect; that is, an account may begin following a media account if:
\begin{enumerate}
    \item another account it follows also follows the media account
    \item the media account shares content sufficiently close to its own ideology
\end{itemize}

Suppose that a non-media account $i$ follows another non-media account $j$, which in turn follows the media account $m$.


\subsection{Adding noise to ideology}

Most larger media outlets do not have a unique ideological voice; different authors or editors may choose to write or share content varying across a range of ideologies.


\subsection{Smooth cutoff functions}

In this section we address a generalization of the receptiveness function that does not use a step function cutoff to determine receptiveness.\footnote{Certainly, this author would prefer to believe that people may be occasionally receptive to sufficiently high-quality arguments, even if they do not match the listener's existing ideology.}

\subsection{Media networks}

\subsection{State actors}

\section{Data}

\section{Simulation results}

\section{Discussion}

\end{document}