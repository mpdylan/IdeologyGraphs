\documentclass[10pt]{amsart}
\usepackage{amsmath}
\usepackage{amssymb}
\usepackage{graphicx}
\usepackage{multicol}
\usepackage{tikz}

\title{}
\author{Dylan Murphy}
\institution{University of Arizona School of Information\footnote{1103 E. 2nd St., Tucson, Arizona, 85721, USA}}
\date{\today}

\begin{document}

\maketitle

\begin{abstract}
The propagation of informatic phenomena such as ideology, sentiment, and disinformation across social networks such as Facebook or Twitter has been studied mathematically and computationally by many scholars, including researchers at Facebook itself, in some detail over the past decade.
The expansion of social media as a marketing and political tool used by corporations, politicians, and state actors makes this a topic of practical social concern as well as a topic of mathematical interest.
Recently, Brooks and Porter\cite{HZBMAP2020} described a dynamical systems model for the evolution of ideology on directed social networks in the presence of media accounts.
Their model reproduced empirically observed effects such as the concentration of accounts around influential media sources and the formation of so-called ``echo chambers.''
In this article, we consider several expansions of the Brooks-Porter model.
\end{abstract}

\section{Introduction}



\section{Existing models}

In this model, a social network is represented by a directed graph.
The vertices of this graph represent accounts; the edges represent the follower relation.
Note that in real social networks, the relation may be asymmetric (e.g., Twitter, where accounts \emph{follow} one another) or symmetric (e.g., Facebook, where accounts typically are \emph{friends}).

\subsection{Media accounts}

A key feature of the Brooks-Porter model is the presence of media accounts, which are accounts that do not follow any other accounts.
Clearly, media accounts are static ``sources'' of ideology; since they do not follow other accounts, their state variables remain constant in time.

\section{Expanded models}

\subsection{Nonlinear updating}

\subsection{Smooth cutoff functions}

\subsection{Media networks}

\subsection{State actors}

\section{Data}

\section{Simulation results}

\section{Discussion}

\end{document}