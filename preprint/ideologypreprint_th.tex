\documentclass[10pt]{amsart}
\usepackage{amsmath}
\usepackage{amssymb}
\usepackage{graphicx}
\usepackage{multicol}
\usepackage{tikz}

\title{You don't know what you don't know: self-location in a dynamical model of ideology on social networks}
\author{Dylan Murphy}
\institution{University of Arizona School of Information\footnote{1103 E. 2nd St., Tucson, Arizona, 85721, USA}}
\date{\today}

\begin{document}

\maketitle

\begin{abstract}

\end{abstract}

\section{Introduction}



This article is a companion to a simulation-based investigation of several generalizations of the Brooks-Porter model.

\section{Existing models}



In this model, a social network is represented by a directed graph.
The vertices of this graph represent accounts; the edges represent the follower relation.
Note that in real social networks, the relation may be asymmetric (e.g., Twitter, where accounts \emph{follow} one another) or symmetric (e.g., Facebook, where accounts typically are \emph{friends}).

\subsection{Media accounts}

A key feature of the Brooks-Porter model is the presence of media accounts, which are accounts that do not follow 

\section{Expanded models}

\subsection{Nonlinear updating}

\subsection{Smooth cutoff functions}

\subsection{Media networks}

\subsection{State actors}

\section{Data}

\section{Simulation results}

\section{Discussion}

\end{document}